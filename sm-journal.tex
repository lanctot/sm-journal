%% This is file `elsarticle-template-1-num.tex',
%%
%% Copyright 2009 Elsevier Ltd
%%
%% This file is part of the 'Elsarticle Bundle'.
%% ---------------------------------------------
%%
%% It may be distributed under the conditions of the LaTeX Project Public
%% License, either version 1.2 of this license or (at your option) any
%% later version.  The latest version of this license is in
%%    http://www.latex-project.org/lppl.txt
%% and version 1.2 or later is part of all distributions of LaTeX
%% version 1999/12/01 or later.
%%
%% The list of all files belonging to the 'Elsarticle Bundle' is
%% given in the file `manifest.txt'.
%%
%% Template article for Elsevier's document class `elsarticle'
%% with numbered style bibliographic references
%%
%% $Id: elsarticle-template-1-num.tex 149 2009-10-08 05:01:15Z rishi $
%% $URL: http://lenova.river-valley.com/svn/elsbst/trunk/elsarticle-template-1-num.tex $
%%
\documentclass[preprint,12pt]{elsarticle}

%% Use the option review to obtain double line spacing
%% \documentclass[preprint,review,12pt]{elsarticle}

%% Use the options 1p,twocolumn; 3p; 3p,twocolumn; 5p; or 5p,twocolumn
%% for a journal layout:
%% \documentclass[final,1p,times]{elsarticle}
%% \documentclass[final,1p,times,twocolumn]{elsarticle}
%% \documentclass[final,3p,times]{elsarticle}
%% \documentclass[final,3p,times,twocolumn]{elsarticle}
%% \documentclass[final,5p,times]{elsarticle}
%% \documentclass[final,5p,times,twocolumn]{elsarticle}

%% if you use PostScript figures in your article
%% use the graphics package for simple commands
%% \usepackage{graphics}
%% or use the graphicx package for more complicated commands
%% \usepackage{graphicx}
%% or use the epsfig package if you prefer to use the old commands
%% \usepackage{epsfig}

%% The amssymb package provides various useful mathematical symbols
\usepackage{graphicx}
\usepackage{amssymb}
\setcounter{tocdepth}{3}
\usepackage{amsmath}
\usepackage{amssymb}
\usepackage{color}
\usepackage{tikz}
\usepackage{pgfplots}
\usepackage{caption}
\usepackage{subcaption}
\usepackage{comment}
\usepackage{booktabs}
\usepackage{nopageno}

\usepackage[algo2e, noend, noline, linesnumbered]{algorithm2e}
%\DontPrintSemicolon

\makeatletter
\newcommand{\pushline}{\Indp}% Indent
\newcommand{\popline}{\Indm}
\makeatother
\DeclareMathOperator{\pess}{pess}
\DeclareMathOperator{\opti}{opti}
\newcommand{\argmax}{\operatornamewithlimits{argmax}}

\captionsetup{compatibility=false}



%\pgfplotsset{compat=newest}
\usetikzlibrary{arrows,shapes,petri}

\newcommand{\bE}{\mathbb{E}}
\newcommand{\cA}{\mathcal{A}}
\newcommand{\cC}{\mathcal{C}}
\newcommand{\cD}{\mathcal{D}}
\newcommand{\cI}{\mathcal{I}}
\newcommand{\cN}{\mathcal{N}}
\newcommand{\cO}{\mathcal{O}}
\newcommand{\cS}{\mathcal{S}}
\newcommand{\cT}{\mathcal{T}}
\newcommand{\cZ}{\mathcal{Z}}
\newcommand{\eg}{{\it e.g.,}~}
\newcommand{\ie}{{\it i.e.,}~}

\definecolor{darkgreen}{RGB}{0,125,0}
\newcounter{mwNoteCounter}
\newcounter{mlNoteCounter}
\newcounter{vlNoteCounter}
\newcounter{bbNoteCounter}
\newcommand{\mwinands}[1]{{\small \color{blue} $\blacksquare$ \refstepcounter{mwNoteCounter}\textsf{[RGG]$_{\arabic{mwNoteCounter}}$:{#1}}}}
\newcommand{\mlanctot}[1]{{\small \color{darkgreen} $\blacksquare$ \refstepcounter{mlNoteCounter}\textsf{[ML]$_{\arabic{mlNoteCounter}}$:{#1}}}}
\newcommand{\vlisy}[1]{{\small \color{red} $\blacktriangle$ \refstepcounter{vlNoteCounter}\textsf{[VL]$_{\arabic{vlNoteCounter}}$:{#1}}}}
\newcommand{\bbosansky}[1]{{\small \color{orange} $\blacktriangle$ \refstepcounter{bbNoteCounter}\textsf{[BB]$_{\arabic{bbNoteCounter}}$:{#1}}}}

\newtheorem{theorem}{Theorem}[section]
\newtheorem{lemma}[theorem]{Lemma}
\newtheorem{proposition}[theorem]{Proposition}
\newtheorem{corollary}[theorem]{Corollary}

\newenvironment{proof}[1][Proof]{\begin{trivlist}
\item[\hskip \labelsep {\bfseries #1}]}{\end{trivlist}}

%% The amsthm package provides extended theorem environments
%% \usepackage{amsthm}

%% The lineno packages adds line numbers. Start line numbering with
%% \begin{linenumbers}, end it with \end{linenumbers}. Or switch it on
%% for the whole article with \linenumbers after \end{frontmatter}.
%% \usepackage{lineno}

%% natbib.sty is loaded by default. However, natbib options can be
%% provided with \biboptions{...} command. Following options are
%% valid:

%%   round  -  round parentheses are used (default)
%%   square -  square brackets are used   [option]
%%   curly  -  curly braces are used      {option}
%%   angle  -  angle brackets are used    <option>
%%   semicolon  -  multiple citations separated by semi-colon
%%   colon  - same as semicolon, an earlier confusion
%%   comma  -  separated by comma
%%   numbers-  selects numerical citations
%%   super  -  numerical citations as superscripts
%%   sort   -  sorts multiple citations according to order in ref. list
%%   sort&compress   -  like sort, but also compresses numerical citations
%%   compress - compresses without sorting
%%
%% \biboptions{comma,round}

% \biboptions{}


\journal{Artificial Intelligence}

\begin{document}

\begin{frontmatter}

%% Title, authors and addresses

%% use the tnoteref command within \title for footnotes;
%% use the tnotetext command for the associated footnote;
%% use the fnref command within \author or \address for footnotes;
%% use the fntext command for the associated footnote;
%% use the corref command within \author for corresponding author footnotes;
%% use the cortext command for the associated footnote;
%% use the ead command for the email address,
%% and the form \ead[url] for the home page:
%%
%% \title{Title\tnoteref{label1}}
%% \tnotetext[label1]{}
%% \author{Name\corref{cor1}\fnref{label2}}
%% \ead{email address}
%% \ead[url]{home page}
%% \fntext[label2]{}
%% \cortext[cor1]{}
%% \address{Address\fnref{label3}}
%% \fntext[label3]{}

\title{Algorithms for Computing Strategies in Two-Player Simultaneous Move Games}

%% use optional labels to link authors explicitly to addresses:
%% \author[label1,label2]{<author name>}
%% \address[label1]{<address>}
%% \address[label2]{<address>}

\author{}

\address{}

\begin{abstract}
Test.
\end{abstract}

\begin{keyword}
%% keywords here, in the form: keyword \sep keyword

%% MSC codes here, in the form: \MSC code \sep code
%% or \MSC[2008] code \sep code (2000 is the default)

\end{keyword}

\end{frontmatter}

%%
%% Start line numbering here if you want
%%
% \linenumbers
%% The Appendices part is started with the command \appendix;
%% appendix sections are then done as normal sections
%% \appendix

%% \section{}
%% \label{}

%% References
%%
%% Following citation commands can be used in the body text:
%% Usage of \cite is as follows:
%%   \cite{key}          ==>>  [#]
%%   \cite[chap. 2]{key} ==>>  [#, chap. 2]
%%   \citet{key}         ==>>  Author [#]

%% References with bibTeX database:

%% main text
\section{Introduction}
\label{sec:intro}

Strategic decision-making in multiagent environments is an imporant problem in artificial intelligence. 
With the growing number of automated agents interacting with humans and with each other, the need to 
understand these strategic interactions at a fundamental level is becoming increasingly important. 
Today, agent interactions occur in many diverse situations, such as e-commerce, social networking, and 
general-purpose robotics, each of which create complex problems that arise from conflicting agent 
preferences. 

Much research has been devoted to developing algorithms that reasoning about or learn multistep 
interactions. For example, adversarial search has been a central topic of artificial intelligence 
since the inception of the field itself, leading to very strong rational behavior in 
Chess~\cite{Campbell02deepblue}. Advances in machine learning for multistep interactions 
(\eg reinforcement learning) have led to self-play algorithms that can achieve master level play 
in Backgammon~\cite{Tesauro95TDGammon}. 

The most common model for these multistage environments is one with strictly {\it sequential} 
iteractions. That is, each player chooses an action when it is their turn to act, and the environment 
generates a new state, with the next player's turn to act, and so on. This model is sufficient in many 
settings, such as Chess and Backgammon, but is not a good representation of the environment when agents 
are allowed to act simultaneously, such as in real-world situations like auctions and autonomous driving. 

In this paper, we present several algorithms for computing strategies in adversarial settings where
players are allowed to act simultaneously. We cover both the offline case, where computation time is 
abundant and strategies are precomputed, and the online case, where computation time is limited and 
agents must search online. We are concerned both with the quality of strategies based on 
their worst-case performance in theory and their obversed performance in practice. We compare and 
contrast the algorithms and choices in the offline and online cases, and thoroughly evaluate each 
algorithm on a suite of games. 

There has been a number of algorithms designed for simultaneous move games that can be classified into three categories: 
(1) iterative learning algorithms, 
(2) exact backard induction algorithms, 
(3) approximative sampling algorithms.
The first type computes strategies through iterated self-play.
The second type computes a Nash equilibrium strategy of the game. 
The third type computes strategies by approximating utilities using sampling. 

\subsection{Iterative Learning Algorithms}

% mlanctot: I put this first because I thought historically it kind of made more sense.
%           Also, then it kind of leads into the ones that we will look at. 

% http://students.cs.byu.edu/~cs670ta/Fall2009/MinimaxQLearning.pdf
% http://www.ualberta.ca/~szepesva/papers/ml96.ps.pdf
% http://www.cogsci.rpi.edu/~rsun/si-mal/article3.pdf
% http://www.cs.duke.edu/~parr/uai2002.ps.gz

A significant amount of interest in simultaneous move games was generated by initial work 
on multiagent reinforcement learning. In multiagent reinforcement learning, each agent acts simultaneously and 
the joint action determines how the state changes. Littman introduced Markov games to model these interactions 
as well as a variant of Q-learning called Minimax-Q to compute strategies~\cite{Littman94markovgames,Littman01Value}.
Minimax-Q modifies the learning rule so that the value of the next state (the subgame) is obtained by solving
a linear program using the estimated values of that subgame's root.
As is common in these settings, the goal of each agent is to maximize their expected utility. 
In two-player zero-sum Markov games, an optimal policy corresponds to a Nash equilibrium strategy, which assures the agent 
the highest worst-case expected payoff. Initial results provided conditions under which approximate dynamic 
programming could be used to guarantee convergence to the optimal value function and 
policies~\cite{Littman96ageneralized}. Later, in \cite{Lagoudakis02}, Lagoudakis \& Parr provided stronger bounds 
and convergence guarantees for least squares temporal different learning using linear function approximation. 

At around this same time period, gradient ascent methods were introduced for playing 
repeated games~\cite{Singh20Nash,Bowling01WoLF}. These algorithms update strategies in a direction of the strategy space 
that increases expected payoff with respect to the opponent's strategy. These were then generalized and combined, and 
shown to minimize regret over time~\cite{Zinkevich03Online,Bowling05Convergence}, leading to strong convergence 
guarantees in multiagent learning. More no-regret algorithms followed and were applied to an imperfect information 
games in sequence-form (One-Card Poker)~\cite{Gordon06No}. Soon later, counterfactual regret (CFR) minimization was 
introduced for large imperfect information games~\cite{CFR}. CFR has gained much attention due to its success in 
computing Poker AI strategies, and in this paper we analyze the effectiveness of a specific form of Monte Carlo CFR 
for the first time in simultaneous move games. 

As we focus on zero-sum simultaneous move games in this paper, the work on multiagent learning in general-sum and 
cooperative games has been omitted. A modern survey of the relevant previous work in multiagent 
reinforcement learning and game theory (including the zero-sum case) is presented in~\cite{Nowe12MARLchapter}. 

\subsection{Exact Backward Induction Algorithms}

The techniques in this section are based on the backward induction algorithm (cf. \cite{Shoham09}), 
a form of dynamic programming~\cite{Bellman57} often presented for purely sequential games. 
A slightly modified variant of the algorithm can also be applied to simultaneous move 
games (e.g., see \cite{Ross71Goofspiel,buro2003,Rhoads12Computer}). 
%\bbosansky{We do need some better reference here.}) 
% mlanctot: added the Ross 1971 paper; as far as I know it's the earliest one. Would be nice to have a text book cover this.
The algorithm searches through the game tree in the depth-first manner and after computing the values of all the succeeding subgames, it solves the normal-form game corresponding to this state (i.e., computes a NE of the matrix game in the current state of the game), and propagates the calculated game value to the predecessor. The result of the backward-induction algorithm is a refinement of NE called \emph{subgame-perfect Nash equilibrium}. 

There are two notable algorithms that improve the standard backward induction in simultaneous move games. 
First is an algorithm by Saffidine et al.~\cite{Saffidine12SMAB} termed simultaneous move alpha-beta algorithm (SMAB). 
The main idea of the algorithm is to reduce the number of the recursive calls of the backward-induction algorithm by removing dominated actions in every stage game. The algorithm keeps bounds on the utility value for each successor in a game state. 
The lower and upper bounds represent the threshold values, for which neither of the actions of the player is dominated by any other action in the current matrix game. These bounds are calculated by linear programs in the state given existing exact values (or appropriate bounds) of the utility values of all the other successors of the state. If they form an empty interval (the lower bound is higher than the upper bound), pruning takes place and the dominated action is no longer considered in this state afterward. 
SMAB outperforms classical backward induction, however, the computational speed-up is only marginal. 

% mlanctot: This seems overly negative toward SMAB and I don't think we need to justify ourselves at this point in the paper. 
%           How about this.. in the evaluation section (next to the more impressive DOAB result) let's put something like:
%           Note: this saving is several orders of magnitude more than reported by SMAB~\cite{Saffidine12SMAB}. 
%SMAB outperforms classical backward induction, however, the computational speed-up is limited. 
%The reason is that computing dominated actions is a costly operation that does not prune many actions; hence, much of the game tree is still fully evaluated. The authors propose further enhancements in their work~\cite{Saffidine12SMAB}, however, these improvements are domain-specific heuristics and do not significantly change the overall performance of the algorithm. Since other exact algorithms achieve computation speed-up in several orders of magnitude compared to backward induction (as we show in Section~\ref{sec:eval}), we do not explicitly use SMAB in our experiments.
% bbosansky: Ok, I agree.

The second exact algorithm that is significantly faster compared to the classical backward induction was introduced in~\cite{Bosansky13Using}.
This algorithm is described in detail in Subsection~\ref{sec:doab}. The main idea is to integrate two key components: (1) instead of evaluating all successors in each state of the game and solving a normal-form game, the algorithm exploits the iterative framework known in game theory as double-oracle algorithm~\cite{McMahan03Planning}; (2) the algorithm computes bounds on the utility values of the successors by serializing the subgames and running classical alpha-beta algorithm. 

Finally, since simultaneous move games can be seen as standard extensive-form games with imperfect information, one can use techniques 
designed for large imperfect information games. 
An algorithm that is also built on double-oracle is the Range-of-Skill algorithm~\cite{Zinkevich07New}.  
However, the number of iterations required by this algorithm in the worst case can be large~\cite{Hansen08On}. 
There are also state-of-the-art algorithms for solving generic extensive-form games with imperfect information, based on sequence-form 
optimization problems~\cite{koller1996,Sandholm10The,bosansky2013-aamas}. 
However, these algorithms do not exploit the specific structure of simultaneous move games and could require memory that is linear 
in the size of the game tree. In practice, this prohibits scaling to larger games (see, \eg \cite{Saffidine12SMAB}) and causes weak performance
compared to tailored algorithms.

%\mlanctot{We should mention the Range-of-Skill algorithm \cite{Zinkevich07New} somewhere in this subsection. It was shown that there are very bad worst cases even for simple games \cite{Hansen08On}.. do we know if such bad cases are less likely to occur in simultaneous move games? It'd be nice if we could say anything at all on this.}\bbosansky{Not exactly sure how to approach this ... I have never really understood the benefits of ROS well.}

\subsection{Approximative Sampling Algorithms} \label{sec:related:sampling}

Monte Carlo Tree Search (MCTS) is a simulation-based state space search technique often used in extensive-form games \cite{Coulom06,UCT}. 
Having first seen practical success in computer Go \cite{Gelly2011,Gelly12}, MCTS has since been applied successfully to simultaneous move games and imperfect information games as well~\cite{Ciancarini10Kriegspiel}. 
Most of the successful applications use classical exploration/exploitation formula Upper Confidence Bounds (UCB)~\cite{UCB} as a selection strategy. These variants of MCTS are also known as UCT (UCB applied to trees). The first application of MCTS to simultaneous move games was in general game playing (GGP) \cite{GGP} programs: {\sc Cadiaplayer} \cite{Cadiaplayer,Finnsson12} uses UCB selection strategy for each player in a single game tree. The success of MCTS algorithm was demonstrated by success of {\sc Cadiaplayer} which was the top-ranked player of the GGP competition between 2007 and 2009, and also in 2012.

Despite this success, Shafiei et al. in \cite{Shafiei09} provide a counter-example showing that this straightforward application of UCT does not
converge to NE even in the simplest simultaneous move games and that a player playing a NE can exploit this strategy. Another variant of UCT, which has been applied to  Tron~\cite{Samothrakis10Tron}, builds the tree as if the players were moving sequentially giving one of the players an informational advantage. This approach also cannot converge to NE in general. For this reason, other variants of MCTS were considered for simultaneous move games. Teytaud and Flory describe a search algorithm for games with short-term imperfect information~\cite{Teytaud11Upper}, which are a generalization of simultaneous move games. Their algorithm uses a different selection strategy, called Exp3~\cite{Auer2003Exp3},
and was shown to work well in the Internet card game Urban Rivals. We provide details of these two main existing selection functions in Subsections~\ref{sec:duct} and~\ref{sec:exp3}.
A more thorough experimental investigation of different selection policies including UCB, UCB1-Tuned, UCB1-greedy, Exp3, and more is reported in the game of Tron \cite{Perick12Comparison}. The work by Lanctot et al.~\cite{Lanctot13Goofspiel} compares some of these variants and proposes Online Outcome Sampling, a search version of Monte Carlo CFR~\cite{Lanctot09Sampling}, which computes an approximate equilibrium strategy with high probability. We describe this algorithm in Subsection~\ref{sec:oos}.
Finally, Lisy et al.~\cite{lisy2013-nips} present variants of MCTS that provably converge to Nash equilibria in simultaneous move games, in general, using any regret-minimizing algorithm at each stage, showing observed worst-case behavior in several cases. 

There have been two recent studies that examine the head-to-head performance of these variants in practice. 
The first builds on previous work in Tron~\cite{Lanctot13Tron} by varying the shape of the initial board, 
comparing previous serialized variants of simultaneous move MCTS. The authors found that UCB1-Tuned worked 
particularly well in Tron when using knowledge-based playout policies. The success of UCB1-Tuned differed in 
a similar study of the same variants across nine domains~\cite{Tak14smmcts} without domain knowledge. In this 
work, the chosen games were ones inspired by previous work in general game playing and did not include chance elements. 
Results indicate that parameter-tuning landscapes do not seem as smooth as in the purely sequential case. 

% mlanctot: said enough about this paper already...
%Generally, head-to-head performance of UCB variants perform well, despite their theoretical shortcomings, with Oshi Zumo and 
%Goofspiel being notable exceptions where MCTS using a regret matching selection policy performs particularly well.

\subsubsection{Simulation-based Search in Real-time Games}

%\mlanctot{Brief overview of relevant work in RTS/video games}

Real-time games are not turn-based and represent a realistic physical situations where agents can move freely in space. 
The state of the game is a continuous function of time and the effect of some actions may only be realized some time 
after the decision is made. These games are often appropriately modeled as a simultaneous move game with very short 
delays (40 milliseconds) between frames. 

MCTS has enjoyed some success in these types of games, in the single-agent 
setting~\cite{Pepels14Monte,Perez14PTSP} and multiagent setting~\cite{Balla09UCT}, much of this work inspired by video 
games~\cite{Cowling13Video,BellemareNVB13,Ontanon13RTSSurvey}. Few of these works have considered MCTS
in the simultaneous move game directly. 
In one of the first papers on real-time strategy games, the authors used randomized serialization 
of the game~\cite{kovarsky2005heuristic}, or strategy simulation
from scripts was used to build a single matrix of values from which an equilibrium strategy was 
computed using linear programming~\cite{Sailor07adversarial}.  
This search can be extended to multiple nodes where internal nodes would correspond to scripts being interrupted to replan, similarly to \cite{lisy2009gbgts}.
MCTS-style multistage replanning was also applied to a real-time battle scenario which was also accurately
represented as a discrete simultaneous move game~\cite{Beard12Using}. Results of this work show and that multistage
forward replanning can improve upon single-stage forward planning, and can produce approximate Nash equilibrium strategies
when mixed strategies are computed at each stage during search.
Around the same time, a serialized (sequential) version of alpha-beta was proposed for simultaneous move games 
and run on combat scenarios~\cite{Churchill2012Fast}; this algorithm is described in greater detail in
Section~\ref{sec:algs:biab} as it forms the basis of the follow-up enhanced by
double-oracle, presented in Section~\ref{sec:algs:doab}.

In this paper, we focus on the analysis of different algorithms for two-player simultaneous move games. Therefore, 
problems arising from discrete modeling of continuous time and space remain outside the scope of this paper.

%The benefit of serializing the game is that bounds
%on the correct minimax value can be obtained for the underlying simultaneous move game. These bounds can be used to cut out
%parts of the game tree in backward induction algorithms, and is explained in detail in Section~\ref{sec:algs:biab}.







\section{Simultaneous Move Games \label{sec:smg}}

A finite game with simultaneous moves and chance can be described by a tuple 
$(\cN, \cS = \cD\cup\cC\cup\cZ, \cA, \cT, \Delta_c, u_i, s_0)$.
The player set $\cN = \{ 1, 2, c \}$ contains player labels, where 
$c$ denotes the chance player and by convention a player is denoted $i \in \cN$.
$\cS$ is a set of states, with $\cZ$ denoting the terminal states, $\cD$ the states where players make decisions, 
and $\cC$ the possibly empty set of states where chance events occur. $\cA = \cA_1 \times \cA_2$ is the set of 
joint actions of individual players. We denote $\cA_i(s)$ the actions available to player $i$ in state $s \in \cS$. 
The transition function $\cT : \cS \times \cA_1 \times \cA_2 \mapsto \cS$ defines the successor state given a current 
state and actions for both players. $\Delta_c:\cC \mapsto \Delta(\cS)$ describes a probability distribution over 
possible successor states of the chance event. The utility functions 
$u_i : \cZ \mapsto [v_{\min}, v_{\max}] \subseteq \mathbb{R}$ gives the utility of player $i$, with 
$v_{min}$ and $v_{\max}$ denoting the minimum and maximum possible utility respectively. We assume constant-sum 
games: $\forall z \in \cZ, u_1(z) = k - u_2(z)$. 
The game begins in an initial state $s_0$. 
\bbosansky{redefine stochastic successors $\Delta_c:\cS \mapsto \Delta(\cS)$ (identity for non-chance nodes)}
\bbosansky{define value of the matrix games}

\begin{figure}[b!]
\centering
\begin{subfigure}{12cm}
\centering
\includegraphics[width=6.0cm]{figures/tree} \hspace{0.03cm} \includegraphics[width=5.5cm]{figures/goof3}\\
\end{subfigure}%\\
\caption{Examples of a two-player simultaneous game without chance nodes (left) which has Matching Pennies as a 
subgame, and a portion of 3-card Goofspiel including chance nodes (right).
The dark squares are terminal states. The values shown are optimal values that could be obtained by backward induction.\\
\label{fig:example}}
\end{figure}

A {\it matrix game} is a single step simultaneous move game with action sets $\cA_1$ and $\cA_2$. 
Each entry in the matrix $A_{rc}$ where $(r,c) \in A_1 \times A_2$ corresponds to a payoff (to player 1) if row $r$ is 
chosen by player 1 and column $c$ by player 2. 
For example, in Matching Pennies, each player has two actions (heads or tails). The row player receives a payoff of 1 
if both players choose the same action and 0 if they do not match. 
Two-player simultaneous move games are sometimes called {\it stacked matrix games} because at every state 
$s$ there is a joint action set $\cA_1(s) \times \cA_2(s)$ that either leads to a terminal state or (possibly after a 
chance transition) to a subgame which is itself another stacked matrix game. 

A {\it behavioral strategy} for player $i$ is a mapping from states $s \in \cS$
to a probability distribution over the actions $\cA_i(s)$, denoted $\sigma_i(s)$. 
Given a profile $\sigma = (\sigma_1, \sigma_2)$, define the probability of reaching a terminal state $z$ under $\sigma$ as 
$\pi^\sigma(z) = \pi_1(z) \pi_2(z) \pi_c(z)$, where each $\pi_i(z)$ is a product of probabilities of the actions taken 
by player $i$ along the path to $z$ ($c$ being chance's probabilities). Define $\Sigma_i$ to be the set of behavioral 
strategies for player $i$. A Nash equilibrium profile in this case is a pair of behavioral strategies optimizing
\begin{equation}\label{eq:ne}
V^* = \max_{\sigma_1 \in \Sigma_1} \min_{\sigma_2 \in \Sigma_2} \bE_{z \sim \sigma}[u_1(z)]
   = \max_{\sigma_1 \in \Sigma_1} \min_{\sigma_2 \in \Sigma_2} \sum_{z \in Z} \pi^\sigma(z) u_1(z).
\end{equation}
In other words, none of the players can improve their utility by deviating unilaterally. 
For example, the Matching Pennies matrix game has a single state and the only equilibrium strategy is to mix equally between 
both actions, \ie play with a {\it mixed strategy} (distribution) of $(0.5, 0.5)$ giving an expected payoff of $V^* = 0.5$. 
If the strategies also optimize Equation \ref{eq:ne} in each subgame starting in an arbitrary state, the equilibrium strategy 
is termed subgame perfect.

In two-player constant sum games a (subgame perfect) Nash equilibrium strategy is often considered to be optimal. It guarantees 
the payoff of at least $V^*$ against any opponent. Any non-equilibrium strategy has its nemesis, which will make it win less 
than $V^*$ in expectation. Moreover, subgame perfect NE strategy can earn more than $V^*$ against weak opponents. After the 
opponent makes a sub-optimal move, the strategy will never allow it to gain the loss back.
The value $V^*$ is known as the minimax-optimal value of the game
and is the same for every equilibrium profile by von Neumann's minimax theorem.

A two-player simultaneous move game is a specific type of two-player imperfect information extensive-form game. 
In imperfect information games, states are grouped into {\it information sets}: two states $s, s' \in I$ if the player 
to act at $I$ cannot distinguish which of these states the game is currently in. Any simultaneous move game can be modeled 
using an information set to represent a half-completed transition, \ie $\cT(s, a_1, ?)$ or $\cT(s, ?, a_2)$. 

The model described above is similar to a two-player finite horizon Markov Game~\cite{Littman94markovgames} with chance 
events. Examples of such games are depicted in Figure~\ref{fig:example}. 

\section{Offline Strategy Computation}

This section focuses on problems of offline solving simultaneous-move games. 
The baseline algorithm for solving simultaneous-move games is backward induction~(Section~\ref{sec:algs:bi}).
Therefore, we start our description of the algorithms focusing on algorithms based on the backward induction, then we introduce algorithms based on no-regret learning and Monte Carlo sampling.
After formally describing the backward induction we present a modification that exploits quick calculation of upper and lower bounds in a simultaneous-move game~(Section~\ref{sec:algs:biab}).
Then, we further improve the algorithm by speeding up the calculation of NE in stage games, exploiting the iterative framework known as the double-oracle algorithm~(Seciton~\ref{sec:algs:doab}).
\bbosansky{CFR/MCTS introduction}

\subsection{Backward Induction}\label{sec:algs:bi}
Standard backward induction algorithm is based on the depth-first search that in each state of the game evaluates all successors, creates matrix game for the current state, solves the matrix game, and propagates back the value of the matrix game. The pseudocode of the algorithm is depicted in Algorithm~\ref{alg:backwardinduction}. When there is a chance node succeeding the current state $s$, the algorithm directly evaluates all successors of this chance node calculating an expected utility: the value of each subgame rooted in node $s'$ calculated by recursive call is weighted by the probability of the stochastic transition $\Delta_{\cT(s,r,c)}$~(line~\ref{alg:bi:recursive}).

\begin{algorithm2e}[t]
\small
\SetKwInOut{Input}{input}\SetKwInOut{Output}{output}
\Input{$s$ -- current matrix game; $i$ -- searching player}
\If{$s \in \cZ$} {\Return $u_i(s)$} \label{alg:bi:stop1}
\For{$r \in \cA_1(s)$}{
	\For{$c \in \cA_2(s)$} {
	$A_{rc} \leftarrow \sum_{s' \in S \;:\; \Delta_{\cT(s,r,c)}(s') > 0} \Delta_{\cT(s,r,c)}(s')\cdot \textrm{BI}(s',i)$ \label{alg:bi:recursive}\;	
	}
}
$v_s \leftarrow$ solve matrix game $A$\;
\Return $v_s$ \label{alg:bi:stop2}
\caption{Backward Induction.}\label{alg:backwardinduction}
\end{algorithm2e}

Once the algorithm evaluates the value of each possible subgame following the current state $s$, then the matrix game $A$ is complete and the algorithm solves the matrix game $A$ using standard linear program (LP) for solving normal-form games:
\begin{eqnarray}
\max & v_s & \\
\textrm{s.t.} & \sum_{a_i \in \cA_{i}}A_{a_i,a_{-i}} \cdot \delta^{s}_i(a_i) \geq v_s & \forall a_{-i} \in \cA_{-i}\\
& \sum_{a_{i} \in \cA_{i}} \delta^{s}_{i}(a_{i}) = 1 \\
& \delta^{s}_{i}(a_{i}) \geq 0 & \forall a_{i} \in \cA_{i} 
\end{eqnarray}
Using the linear programming, the algorithm computes the value of the matrix game, which is propagated to the predecessor~(line~\ref{alg:bi:stop2}). 
If the algorithm evaluates a terminal state, it directly returns the utility value of the state~(line~\ref{alg:bi:stop1}).

\subsection{Backward Induction with Serialized Alpha-Beta Bounds}\label{sec:algs:biab}
Backward induction algorithm can be easily enhanced by using bounds on the value of sub-games. 
These bounds can be calculated very quickly by using a transformation of a simultaneous-move game into a perfect information extensive-form game.
Consider a matrix game representing a simultaneous choice of both players.
This matrix can be serialized by discarding the notion of information sets; hence, letting one player to play first following by the play of the second player. 
The crucial difference between a serialized and a simultaneous-move matrix game is that the second player to move has an advantage of knowing what action has been played in a serialized game.
Therefore, the value of a serialized game where player $i$ is second to move is an upper bound on the value 

\begin{figure}
\includegraphics[width=0.45\textwidth]{figures/serialization1-1.png}
\includegraphics[width=0.45\textwidth]{figures/serialization1-2.png}
\caption{Different serialization of a simultaneous move game.}\label{fig:serialization}
\end{figure}

An example of this serialization is depicted in Figure~\ref{fig:serialization} with the utility value in the leaves depicted only for a single player. 
If this player moves first (the left subfigure), then the value of this serialized game is the lower bound of the value of the game; if this player moves second (the right subfigure), then the value of this serialized game is the upper bound of the value of the game.
Since the serialized games are zero-sum perfect-information games in the extensive form, they can be solved very quickly by using classical AI algorithm such as alpha-beta or negascout.
If the values of both serialized games are equal, then this is equal also to the value of the game. This situation occurs in our example in Figure~\ref{fig:serialization}, where both serialized games have value equal to $3$.

Obtaining these bounds can speed-up the backward induction algorithm. 
Algorithm~\ref{alg:backwardinduction-ab} depicts the pseudocode.
When the backward induction starts evaluating successors of the current state, the algorithm calculates upper and lower bounds using alpha-beta algorithm on serialized variants of the subgame rooted in the successor $s'$ (lines~\ref{alg:biab:ab1}-\ref{alg:biab:ab2}). The serialized game is solved using standard alpha-beta algorithm, where the second player 
If the bounds are equal, the algorithm stores the value (line~\ref{alg:biab:saving}) instead of performing a recursive call~(line~\ref{alg:biab:recursive}).

\begin{algorithm2e}[t]
\small
\SetKwInOut{Input}{input}\SetKwInOut{Output}{output}
\Input{$s$ -- current matrix game; $i$ -- searching player}
\If{$s \in \cZ$} {\Return $u_i(s)$} \label{alg:biab:stop1}
\For{$r \in \cA_1(s)$}{
	\For{$c \in \cA_2(s)$} {
		$A_{rc} \leftarrow 0$\;
		\For{$s' \in S \;:\; \Delta_{\cT(s,r,c)}(s') > 0$} {
		$v^i_{s'} \leftarrow \textrm{alpha-beta}(s',i)$\; \label{alg:biab:ab1}
		$v^{-i}_{s'} \leftarrow \textrm{alpha-beta}(s',-i)$\; \label{alg:biab:ab2}
		\If{$v^{-i}_{s'} < v^i_{s'}$} {
			$A_{rc} \leftarrow A_{rc} + \Delta_{\cT(s,r,c)}(s')\cdot \textrm{BI}\alpha\beta(s',i)$ \label{alg:biab:recursive}\;	
			}
		\Else{
			$A_{rc} \leftarrow A_{rc} + \Delta_{\cT(s,r,c)}(s')\cdot v^i_{s'}$ \label{alg:biab:saving}
			}
		}
	}
}
$v_s \leftarrow$ solve matrix game $A$\;
\Return $v_s$ \label{alg:biab:stop2}
\caption{Backward Induction with Serialized Bounds}\label{alg:backwardinduction-ab}
\end{algorithm2e}

\subsection{Backward Induction with Double Oracle and Serialized Bounds}\label{sec:algs:doab}
Solving a matrix game can be further improved by using iterative double-oracle algorithm~\cite{McMahan03Planning}. 
First of all, we describe the main principles of the double-oracle algorithm in normal-form games, following by describing the application of the algorithm in simultaneous-move game~\cite{Bosansky13Using}.

\subsubsection{Double-oracle Algorithm for Matrix Games}\label{sec:doab}
\begin{figure}
\centering
\includegraphics[width=0.8\textwidth]{figures/DO-scheme}
\caption{Schematic of the double-oracle algorithm for a normal-form game.}\label{fig:do-scheme}
\end{figure}

The main goal of the double-oracle algorithm is to find a solution of a matrix game without a need to construct the complete linear program that solves this game. 
The main idea is to create a restricted game by restricting the players to choose only from a limited set of actions to play.
Then the algorithm iteratively expands the restricted game by allowing the players to choose from new actions.
The new actions are allowed in iterations; each iteration a best response to an optimal strategy of the opponent in the current restricted game is allowed to be played in the restricted game.

Figure~\ref{fig:do-scheme} shows a visualization of the main structure of the algorithm, where the following three steps repeat until convergence:
\begin{enumerate}
\item create a restricted matrix game by limiting the set of actions that each player is allowed to play
\item compute a pair of Nash equilibrium strategies in this restricted game using linear programming
\item for each player, compute a pure best response strategy against the equilibrium strategy of the opponent; pure best response can be \emph{any} action from the original unrestricted game
\end{enumerate}
The best response strategies computed in step 3 are added to the restricted game, the game matrix is expanded by adding new rows and columns, and the algorithm follows with the next iteration. The algorithm terminates if neither of the players can improve the outcome of the game by adding a new strategy to the restricted game; hence, players play best response strategies to the strategy of the opponent. The algorithm maintains the values of expected utilities of the best-response strategies throughout the iterations of the algorithm. These values provide bounds on the value of the original game $v^*$

\subsubsection{Integrating Double-Oracle with Backward Induction}
\begin{algorithm2e}[t]
\small
\SetKwInOut{Input}{input}\SetKwInOut{Output}{output}
\Input{$s$ -- current matrix game; $i$ -- searching player; $\alpha_s,\beta_s$ -- bounds for the game value rooted in state $s$}
\If{$s \in \cZ$} {\Return $u_i(s)$} \label{alg:doab:stop1}
\If{$v_s^{-i} = v_s^i$} {
	\Return $v_s^{i}$
}
initialize $\cA'_i$, $\cA'_{-i}$ with arbitrary actions from $\cA_i, \cA_{-i}$\; \label{alg:doab:init}
\Repeat{$\alpha_s = \beta_s$}{
	\For{$r \in \cA'_i$, $c \in \cA'_{-i}$}{\label{alg:doab:restr1}
		\If{$A'_{rc}$ is not initialized}{
			 $A'_{rc} \leftarrow 0$\;
			\For{$s' \in S \;:\; \Delta_{\cT(s,r,c)}(s') > 0$} {
			$v^i_{s'} \leftarrow \textrm{alpha-beta}(s',i)$\; 
			$v^{-i}_{s'} \leftarrow \textrm{alpha-beta}(s',-i)$\; 
			\If{$v^{-i}_{s'} < v^i_{s'}$} {
				$A'_{rc} \leftarrow A'_{rc} + \Delta_{\cT(s,r,c)}(s')\cdot \textrm{double-oracle}(s',i,v^{-i}_{s'},v^i_{s'})$ \label{alg:doab:recursive}\;	
				}
			\Else{
				$A'_{rc} \leftarrow A'_{rc} + \Delta_{\cT(s,r,c)}(s')\cdot v^i_{s'}$ \label{alg:doab:saving}
				}
			}
		}	
	}
	$\left\langle v_s, \delta' \right\rangle \leftarrow $ solve matrix game $A'$\; \label{alg:doab:NE}
	$\left\langle v^{BR}_i, a^{BR}_{i} \right\rangle \leftarrow $ best-response$(i, \delta'_{-i}, \alpha_s)$\;\label{alg:doab:br1}
	$\left\langle v^{BR}_{-i}, a^{BR}_{-i} \right\rangle \leftarrow $ best-response$(-i, \delta'_{i}, \beta_s)$\; \label{alg:doab:br2}
	$\alpha_s \leftarrow \max(\alpha_s, v^{BR}_i)$,
	$\beta_s \leftarrow \min(\beta_s, v^{BR}_{-i})$\; \label{alg:doab:bounds}
	$\cA'_i \leftarrow \cA'_i \cup \lbrace a^{BR}_i \rbrace$,
	$\cA'_{-i} \leftarrow \cA'_{-i} \cup \lbrace a^{BR}_{-i} \rbrace$ \label{alg:doab:expand}
}
\Return $v_s$ \label{alg:doab:stop2}
\caption{Double Oracle with Serialized Bounds}\label{alg:doab}
\end{algorithm2e}

Double-oracle algorithm for matrix games can be directly incorporated into the backward induction -- instead of immediately evaluating each of the successors of the current game state and solving the linear program, the algorithm can exploit the double-oracle algorithm. Pseudocode in Algorithm~\ref{alg:doab} depicts this integration. In each state of the game, the algorithm initializes the restricted game with an arbitrary action (line~\ref{alg:doab:init}). Afterwards, the algorithm needs to evaluate each of the successors of the restricted game, for which the current value is not known (lines~\ref{alg:doab:restr1}-\ref{alg:doab:saving}). This evaluation is same as for backward induction with alpha-beta algorithm. 

Once all values for the restricted game $A'$ are known, the algorithm solves the restricted game (line~\ref{alg:doab:NE}) and keeps the optimal strategies $\delta'$ of the restricted game. Next, the algorithm calculates best responses for each of the player~(lines~\ref{alg:doab:br1},\ref{alg:doab:br2}), and updates the lower and upper bounds (line~\ref{alg:doab:bounds}). Finally, the algorithm expands the restricted game with best response actions (line~\ref{alg:doab:expand}) until the lower and upper bound are equal. In this case, neither of the best responses improves the current solution from the restricted game; hence, the algorithm has found an equilibrium of the complete unrestricted matrix game corresponding to state $s$.


\begin{algorithm2e}[t]
\small
\SetKwInOut{Input}{input}\SetKwInOut{Output}{output}
\Input{$s$ -- current matrix game; $i$ -- best-response player; $\lambda$ -- bound for the best-response value; $\delta'_{-i}$ -- strategy of the opponent }
$v^{BR}_i \leftarrow \lambda$ \;
$a_i^{BR} \leftarrow \textbf{null}$ \;
\For{$a_i \in \cA_i $\label{alg:br:start}} {%\setminus \cA'_i 
	$v_{a_i} \leftarrow 0$\;
	\For{$a_{-i} \in \cA'_{-i} \;:\; \delta'_{-i}(a_{-i}) > 0$} {\label{alg:br:opp}
		$\lambda_{a_i} \leftarrow \frac{v_i^{BR} - \sum_{a'_{-i} \in \cA'_{-i} \setminus \lbrace a_{-i} \rbrace} \delta'_{-i}(a'_{-i}) \cdot v^i_{\cT(s,a_i,a'_{-i})}}{\delta'_{-i}(a_{-i})}$\; \label{alg:br:bound}
		\If{	$\lambda_{a_i} > v^i_{\cT(s,a_i,a_{-i})}$}{
		continue from line~\ref{alg:br:start} with next $a_i$
		}
		\Else{
		\For{$s' \in S \;:\; \Delta_{\cT(s,a_i,a_{-i})}(s') > 0$} {
%			$v^i_{s'} \leftarrow \textrm{alpha-beta}(s',i)$\; \label{alg:br:ab1}
%			$v^{-i}_{s'} \leftarrow \textrm{alpha-beta}(s',-i)$\; \label{alg:br:ab2}
			\If{$v^{-i}_{s'} < v^i_{s'}$} {
				$v_{a_i,a_{-i}} \leftarrow v_{a_i,a_{-i}} + \Delta_{\cT(s,a_i,a_{-i})}(s')\cdot$ $\textrm{double-oracle}(s',i,v^{-i}_{s'},v^i_{s'})$\label{alg:br:recursive}\;	
				}
			\Else{
				$v_{a_i,a_{-i}} \leftarrow v_{a_i,a_{-i}} + \Delta_{\cT(s,a_i,a_{-i})}(s')\cdot v^i_{s'}$ \label{alg:br:saving}
				}
			}
			$v_{a_i} \leftarrow v_{a_i} + \delta'_{-i}(a_{-i})\cdot v_{a_i,a_{-i}}$\;		
		}
	}
	\If{$v_{a_i} > v_i^{BR}$}{\label{alg:br:max}
		$v_i^{BR} \leftarrow v_{a_i}$\;
		$a_i^{BR} \leftarrow a_i$\label{alg:br:save}
	} 
}
\Return $\langle v_i^{BR}, a_i^{BR} \rangle$ 
\caption{Best Response with Serialized Bounds}\label{alg:br}
\end{algorithm2e}

Next, we describe the algorithm for calculating the best responses. 
The pseudocode of the algorithm is depicted in Algorithm~\ref{alg:br}.
The goal of the algorithm is to find the best action from the original unrestricted game against current strategy of the opponent $\delta'$. 
Throughout the algorithm we use, as before, $v^i_{s'}$ to denote the upper bound of the value of the sub-game rooted in state $s'$ calculated using alpha-beta$(s',i)$. These values are calculated on demand, i.e., they are calculated once needed and cached until the game for state $s$ is not solved.
Moreover, once the algorithm calculates the exact value of a particular subgame, both upper and lower bounds are updated to be equal to actual value of the game. 

The algorithm iteratively tries all actions of player $i$ from the unrestricted game (line~\ref{alg:br:start}). 
Every action $a_i$ is evaluated against the actions of the opponent that are used in the optimal strategy from the restricted game (line~\ref{alg:br:opp}).
Before evaluating against an action of the opponent the algorithm determines, whether the current action of the searching player, $a_i$, can still be the best response action (line~\ref{alg:br:bound}). 
The value $\lambda_{a_i}$ represents the lowest possible expected utility this action must gain against the current action of the opponent $a_{-i}$. 
If this value is strictly higher than the upper bound of the successor (i.e., the value $v^i_{\cT(s,a_i,a_{-i})}$) than the algorithm knows that the action $a_i$ can never be the best response action, and the algorithm proceeds with the next action.
$\lambda_{a_i}$ is calculated by subtracting from the current best response value ($v_i^{BR}$) upper bound of the expected value against all other actions of the opponent ($v^i_{\cT(s,a_i,a'_{-i})}$). Recall, that these values are updated once the algorithm calculates exact values.

If the currently evaluated action $a_i$ can still be the best response, the value of the successor is determined (first by comparing the bounds). Once the expected outcome against all actions of the opponent is evaluated, the expected value of action $a_i$ is compared against the current best-response value (line~\ref{alg:br:max}) and saved, if the expected utility is higher~(line~\ref{alg:br:save}).


\subsection{Simultaneous-Move Monte Carlo Tree Search (SM-MCTS)}\label{sec:smmcts}

Monte Carlo Tree Search (MCTS) is a simulation-based state space search algorithm often used
in game trees. The nodes in the tree represent game states. The main idea is to iteratively run
simulations to a terminal state, incrementally growing a tree rooted at the initial state of the game. In
its simplest form, the tree is initially empty and a single leaf is added each iteration. Each iteration
starts by visiting nodes in the tree, selecting which actions to take based on a selection function and
information maintained in the node. Consequently, the algorithm transitions to a successor state. When a
node is visited whose immediate children are not all in the tree, the node is expanded by adding a
new leaf to the tree. Then, \emph{a rollout policy} (e.g., random action selection) is applied from the new
leaf to a terminal state. The outcome of the simulation is then returned as a reward to the new leaf
and the information stored in the tree is updated.

\begin{algorithm2e}[t]
\small
\SetKwInOut{Input}{input}\SetKwInOut{Output}{output}
\Input{$s$ -- current state of the game; $i$ -- current player}
\If{$s \in \cZ$} {
	\Return $u_i(s)$
}
\If{$s \in T \wedge \exists(a_i, a_{-i}) \in \cA(s)$~not previously visited} {
	choose one of the previously unselected $(a_i, a_{-i})$\;
	$s' \leftarrow \cT(s,a_i,a_{-i})$\;
	$T \leftarrow T \cup \lbrace s' \rbrace$\;
	$v_{s'} \leftarrow $ \emph{\underline{Rollout}}($s'$)\;
	$X_{s'} \leftarrow X_{s'} + v_{s'}$\;
	$n_{s'} \leftarrow n_{s'} + 1$\;
	\emph{\underline{Update}}$(s,a_i,a_{-i},v_{s'})$\;\label{alg:smmcts:up1}
	\Return \emph{\underline{RetVal}}$(v_{s'}, X_{s'}, n_{s'})$ 	
}
\Else{
	$(a_i, a_{-i}) \leftarrow$ \emph{\underline{Select}}$(s)$\;\label{alg:smmcts:select}
	$s' \leftarrow \cT(s,a_i,a_{-i})$\;
	$v_{s'} \leftarrow $ SM-MCTS($s',i$)\;
	$X_{s} \leftarrow X_{s} + v_{s'}$\;
	$n_{s} \leftarrow n_{s} + 1$\;
	\emph{\underline{Update}}$(s,a_i,a_{-i},v_{s'})$\;\label{alg:smmcts:up2}
	\Return \emph{\underline{RetVal}}$(v_{s'}, X_{s}, n_{s})$ 	
}
\caption{Simultaneous Move Monte Carlo Tree Search}\label{alg:smmmcts}
\end{algorithm2e}

Algorithm~\ref{alg:smmmcts} describes a single iteration of SM-MCTS. $T$ represents the MCTS tree in which
each state is represented by one node. Every node $s$ maintains a cumulative reward sum over all
simulations through it, $X_s$, and a visit count $n_s$. Both are initially set to $0$. The critical parts of the
algorithm are the updates on lines~\ref{alg:smmcts:up1} and \ref{alg:smmcts:up2} and the selection on line \ref{alg:smmcts:select}. 
Different algorithms can be used as selection functions (e.g., UCB~\cite{UCB}, Exp3~\cite{Auer2003Exp3}, or regret matching~\cite{Hart00}) and we describe them in the following sections. 
Moreover, different selection functions need different statistics kept in each node; hence, function \emph{\underline{Update}} also differs. 
Finally, the algorithm returns a value based on \emph{\underline{RetVal}} function. 
In our algorithms we consider two different return value functions: the algorithm either directly returns value of the current state $v_{s'}$, or mean value $v_{s'}/n_{s'}$.


\section{Online Search}

\subsection{Iterative Deepening Backward Induction Algorithms}

\subsection{Online Outcome Sampling}

\section{Empirical Evaluation}

\section{Conclusion}


\bibliographystyle{model1-num-names}
\bibliography{sm-journal}

%% Authors are advised to submit their bibtex database files. They are
%% requested to list a bibtex style file in the manuscript if they do
%% not want to use model1-num-names.bst.

%% References without bibTeX database:

% \begin{thebibliography}{00}

%% \bibitem must have the following form:
%%   \bibitem{key}...
%%

% \bibitem{}

% \end{thebibliography}


\end{document}

%%
%% End of file `elsarticle-template-1-num.tex'.
