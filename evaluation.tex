
\subsection{Domains}

In this subsection, we describe the domains used in our experiments.

\begin{description}
\item[Biased Rock, Paper, Scissors] is a payoff-skewed version of the one-shot game Rock, Paper, Scissors shown in 
Figure~\ref{fig:brps}. This game was introduced in \cite{Shafiei09}, and shown that the visit count distribution of 
DUCT does converge to a fixed balanced situation, but not one that 
corresponds to the optimal mixed strategy of $(\frac{1}{16},\frac{10}{16},\frac{5}{16})$. 

\begin{figure}[h!]
\begin{center}
\includegraphics[scale=1.0]{figures/biased-rps}
\end{center}
\caption{Biased Rock, Paper, Scissors matrix game from~\cite{Shafiei09}. \label{fig:brps}}
\end{figure}

\item[Goofspiel] is a card game where each player gets 13 cards marked 1-13, and there is a face down
point-card stack (also 1-13). Every turn, the {\it upcard} (top card of the point-card stack is turned face up,
Each player chooses a {\it bid} card from their hand simultaneously.
The player with the higher bid takes the upcard. The bid cards are then discarded and a new round starts.
At the end of 13 rounds, the player with the highest number of points wins, a tie ends in a draw.

\item[Oshi-Zumo]$(N,K,M)$ is a wrestling simulation game played on a discrete single-dimensional grid with
$2K+1$ positions, where each player starts with $N$ coins~\cite{buro2003}. A wrestler token begins in the middle
position. Every turn,
each player bids $b \ge M$ coins. The coins bid are then discarded and the player bidding the most coins pushes the
wrestler one position closer to the goal for that player. The parameters used in our experiments are $(50,3,0)$.

\item[Pursuit Evasion Games]

\item[Random/Synthetic Games]

\item[Tron] is a two-player game played on discrete grid possibly obstructed by walls. At each
step in Tron both players move to adjacent cells, and a wall is placed in the cells the players started on that turn.
Both players try to survive as long as possible. If both players can only move into a wall, can only move off the board or move into each other at the same turn, the game ends  in a draw. 
\end{description}

\subsection{Offline Equilibrium Computation}

\subsection{Online Search}



