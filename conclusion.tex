
In this paper, we provide an extensive analysis of algorithms for solving and playing zero-sum extensive-form games with perfect information and simultaneous moves. We describe a collection of exact algorithms based on backward induction as well as a collection of Monte Carlo tree search algorithms including our novel algorithms $\doab$, $\biab$, SM-OOS and SM-MCTS with regret matching selection function. 

We empirically compare the performance of these algorithms on six substantially different games in two different settings. In the offline equilibrium computation setting, we show that our novel algorithm based on backward induction, $\doab$, is able to prune large parts of the search space. In most games, $\doab$ is several orders of magnitude faster than the classical backward induction and it is never significantly outperformed by any of its competitors. The only benefit of the sampling algorithms in the offline setting is a rough approximation of the equilibrium solution in a short time. Their results are often inconsistent with short computation times. Given enough time, the results clearly show that SM-OOS achieves the fastest convergence to a Nash equilibrium. Finally, our offline experiments also explained different behavior reported in variants of SM-MCTS with UCT selection function. We have shown that adding \bbosansky{a?}randomization to tie-breaking rules can significantly improve the performance of this algorithm.

The success in the offline equilibrium computation is, however, not a very good indicator of the game playing performance in the online setting of head-to-head matches. First of all, the size of the games used for online experiments is too large for exact algorithms to be applicable without a domain-specific evaluation function. Performance of the representative of the exact algorithms, $\doab$, depends heavily on the accuracy of the used evaluation function. Secondly, in spite of the fastest convergence of SM-OOS among the sampling algorithms, SM-OOS does not always perform well in the online game playing. \reviewchange{This is mainly due to the large variance of the regret updates that increases significantly in these large games. Among the remaining sampling algorithms, SM-MCTS based on regret matching is often very good, but sometimes it was outperformed by SM-MCTS with UCT selection, especially in games that require less randomized strategies.}

%The $\doab$ algorithm cannot work without a domain-specific evaluation function in this setting and its performance heavily depends on its quality. With a good evaluation function, it outperforms the sampling algorithms if they do not use any domain-specific knowledge. However, if even the sampling algorithms are allowed to use the evaluation functions, they always significantly outperform $\doab$. Therefore, we conclude that sampling algorithms are a better choice for game playing in this class of games. 

Our work opens several interesting directions for future research. After introducing a strong pruning algorithm, it is of interest to formally study the limitations of pruning for this class of games, similarly to the theory developed for games with sequential moves. Future work could show if the introduced pruning techniques can be substantially improved or if they are in some sense optimal. \reviewchange{The main prerequisite is, however, estimating the expected number of iterations of the double-oracle algorithms for single step matrix games, which still remains an open problem.}
Furthermore, running large head-to-head tournaments for evaluating the game playing performance is time consuming, quite sensitive to setting correct parameters, and it provides only limited insights into the performance of the algorithms. Proximity to the Nash equilibrium is not always a good indicator of game playing performance; hence, it is interesting to study alternative measures of quality of the algorithms that would better predict their game-playing performance in large games. \\

\noindent {\bf Acknowledgements.} This work is funded by the Czech Science Foundation (grant no. P202/12/2054 and 15-23235S) and the Netherlands
Organisation for Scientific Research (NWO) in the framework of the project Go4Nature, grant number 612.000.938.
Branislav Bo{\v s}ansk{\' y} also acknowledges support from the Danish National Research Foundation and The National Science Foundation of China (under the grant 61361136003) for the Sino-Danish Center for the Theory of Interactive Computation, and the support
from the Center for Research in Foundations of Electronic Markets (CFEM), supported by the Danish Strategic Research Council.
The access to computing and storage facilities owned by parties and projects contributing to the National Grid
Infrastructure MetaCentrum, provided under the
programme ``Projects of Large Infrastructure for Research, Development, and Innovations'' (LM2010005) is highly appreciated.

